\nonstopmode{}
\documentclass[letterpaper]{book}
\usepackage[times,inconsolata,hyper]{Rd}
\usepackage{makeidx}
\makeatletter\@ifl@t@r\fmtversion{2018/04/01}{}{\usepackage[utf8]{inputenc}}\makeatother
% \usepackage{graphicx} % @USE GRAPHICX@
\makeindex{}
\begin{document}
\chapter*{}
\begin{center}
{\textbf{\huge Package `epicR'}}
\par\bigskip{\large \today}
\end{center}
\ifthenelse{\boolean{Rd@use@hyper}}{\hypersetup{pdftitle = {epicR: R Package for Evaluation Platform in COPD}}}{}
\ifthenelse{\boolean{Rd@use@hyper}}{\hypersetup{pdfauthor = {Mohsen Sadatsafavi; Amin Adibi; Kate Johnson}}}{}
\begin{description}
\raggedright{}
\item[Title]\AsIs{R Package for Evaluation Platform in COPD}
\item[Version]\AsIs{0.9.0}
\item[Author]\AsIs{Mohsen Sadatsafavi [aut, cph], Amin Adibi [aut, cre], Kate Johnson [aut]}
\item[Maintainer]\AsIs{Amin Adibi }\email{adibi@alumni.ubc.ca}\AsIs{}
\item[Description]\AsIs{Evaluation Platform in COPD (EPIC) is a Discrete Event Simulation (DES) model that simulates health outcomes of COPD patients based on demographics and individual-level risk factors, based on the model published in Sadatsafavi et al. (2019) <}\Rhref{https://doi.org/10.1177/0272989X18824098}{doi:10.1177/0272989X18824098}\AsIs{>.}
\item[Depends]\AsIs{R (>= 4.1.0)}
\item[License]\AsIs{GPL-3}
\item[Encoding]\AsIs{UTF-8}
\item[Imports]\AsIs{Rcpp (>= 1.0.8), graphics, stats, ggplot2 (>= 3.3.6),
ggthemes, sqldf, jsonlite, readr, reshape2, scales, tidyr,
tibble, dplyr}
\item[LinkingTo]\AsIs{Rcpp, RcppArmadillo}
\item[RoxygenNote]\AsIs{7.3.3}
\item[Suggests]\AsIs{testthat (>= 3.0.0), openxlsx, survival, survminer, knitr,
rmarkdown}
\item[Config/testthat/edition]\AsIs{3}
\item[VignetteBuilder]\AsIs{knitr, rmarkdown}
\item[NeedsCompilation]\AsIs{yes}
\end{description}
\Rdcontents{Contents}
\HeaderA{epicR-package}{\code{epicR} package}{epicR.Rdash.package}
\aliasA{epicR}{epicR-package}{epicR}
%
\begin{Description}
Evaluation Platform in COPD (EPIC) R Package
\end{Description}
%
\begin{Details}
See the README on
\Rhref{https://github.com/resplab/epicR\#readme}{GitHub}
\end{Details}
%
\begin{Author}
\strong{Maintainer}: Amin Adibi \email{adibi@alumni.ubc.ca}

Authors:
\begin{itemize}

\item{} Mohsen Sadatsafavi \email{mohsen.sadatsafavi@ubc.ca} [copyright holder]
\item{} Kate Johnson \email{kate.johnson@alumni.ubc.ca}

\end{itemize}


\end{Author}
\HeaderA{calc\_event\_stack\_size}{Calculate recommended event\_stack\_size for a given number of agents}{calc.Rul.event.Rul.stack.Rul.size}
%
\begin{Description}
Calculate recommended event\_stack\_size for a given number of agents
\end{Description}
%
\begin{Usage}
\begin{verbatim}
calc_event_stack_size(n_agents, time_horizon = 20)
\end{verbatim}
\end{Usage}
%
\begin{Arguments}
\begin{ldescription}
\item[\code{n\_agents}] number of agents to simulate

\item[\code{time\_horizon}] simulation time horizon in years (default 20)
\end{ldescription}
\end{Arguments}
%
\begin{Value}
recommended event\_stack\_size
\end{Value}
\HeaderA{calibrate\_COPD\_inc}{Solves stochastically for COPD incidence rate equation.}{calibrate.Rul.COPD.Rul.inc}
%
\begin{Description}
Solves stochastically for COPD incidence rate equation.
\end{Description}
%
\begin{Usage}
\begin{verbatim}
calibrate_COPD_inc(nIterations = 100, nPatients = 1e+05, time_horizon = 20)
\end{verbatim}
\end{Usage}
%
\begin{Arguments}
\begin{ldescription}
\item[\code{nIterations}] number of iterations for the numberical solution

\item[\code{nPatients}] number of simulated agents.

\item[\code{time\_horizon}] in years
\end{ldescription}
\end{Arguments}
%
\begin{Value}
regression co-efficients as files
\end{Value}
\HeaderA{calibrate\_explicit\_mortality}{Calibrates explicit mortality}{calibrate.Rul.explicit.Rul.mortality}
%
\begin{Description}
Calibrates explicit mortality
\end{Description}
%
\begin{Usage}
\begin{verbatim}
calibrate_explicit_mortality(n_sim = 10^8)
\end{verbatim}
\end{Usage}
%
\begin{Arguments}
\begin{ldescription}
\item[\code{n\_sim}] number of agents
\end{ldescription}
\end{Arguments}
%
\begin{Value}
difference in mortality rates and life table
\end{Value}
\HeaderA{calibrate\_explicit\_mortality2}{Calibrates explicit mortality by Amin}{calibrate.Rul.explicit.Rul.mortality2}
%
\begin{Description}
Calibrates explicit mortality by Amin
\end{Description}
%
\begin{Usage}
\begin{verbatim}
calibrate_explicit_mortality2(n_sim = 10^7)
\end{verbatim}
\end{Usage}
%
\begin{Arguments}
\begin{ldescription}
\item[\code{n\_sim}] number of agents
\end{ldescription}
\end{Arguments}
%
\begin{Value}
difference in mortality rates and life table
\end{Value}
\HeaderA{calibrate\_smoking}{Calibrates smoking}{calibrate.Rul.smoking}
%
\begin{Description}
Calibrates smoking
\end{Description}
%
\begin{Usage}
\begin{verbatim}
calibrate_smoking()
\end{verbatim}
\end{Usage}
%
\begin{Value}
TODO
\end{Value}
\HeaderA{Cget\_agent\_events}{Returns all events of an agent.}{Cget.Rul.agent.Rul.events}
%
\begin{Description}
Returns all events of an agent.
\end{Description}
%
\begin{Usage}
\begin{verbatim}
Cget_agent_events(id)
\end{verbatim}
\end{Usage}
%
\begin{Arguments}
\begin{ldescription}
\item[\code{id}] agent ID.
\end{ldescription}
\end{Arguments}
%
\begin{Value}
all events of agent \code{id}
\end{Value}
\HeaderA{Cget\_agent\_size\_bytes}{Returns the size of agent struct in bytes}{Cget.Rul.agent.Rul.size.Rul.bytes}
%
\begin{Description}
Returns the size of agent struct in bytes
\end{Description}
%
\begin{Usage}
\begin{verbatim}
Cget_agent_size_bytes()
\end{verbatim}
\end{Usage}
%
\begin{Value}
size of agent struct in bytes
\end{Value}
\HeaderA{Cget\_all\_events}{Returns all events.}{Cget.Rul.all.Rul.events}
%
\begin{Description}
Returns all events.
\end{Description}
%
\begin{Usage}
\begin{verbatim}
Cget_all_events()
\end{verbatim}
\end{Usage}
%
\begin{Value}
all events
\end{Value}
\HeaderA{Cget\_all\_events\_matrix}{Returns a matrix containing all events}{Cget.Rul.all.Rul.events.Rul.matrix}
%
\begin{Description}
Returns a matrix containing all events
\end{Description}
%
\begin{Usage}
\begin{verbatim}
Cget_all_events_matrix()
\end{verbatim}
\end{Usage}
%
\begin{Value}
a matrix containing all events
\end{Value}
\HeaderA{Cget\_event}{Returns the events stack.}{Cget.Rul.event}
%
\begin{Description}
Returns the events stack.
\end{Description}
%
\begin{Usage}
\begin{verbatim}
Cget_event(i)
\end{verbatim}
\end{Usage}
%
\begin{Arguments}
\begin{ldescription}
\item[\code{i}] number of event
\end{ldescription}
\end{Arguments}
%
\begin{Value}
events
\end{Value}
\HeaderA{Cget\_events\_by\_type}{Returns all events of a certain type.}{Cget.Rul.events.Rul.by.Rul.type}
%
\begin{Description}
Returns all events of a certain type.
\end{Description}
%
\begin{Usage}
\begin{verbatim}
Cget_events_by_type(event_type)
\end{verbatim}
\end{Usage}
%
\begin{Arguments}
\begin{ldescription}
\item[\code{event\_type}] a number
\end{ldescription}
\end{Arguments}
%
\begin{Value}
all events of the type \code{event\_type}
\end{Value}
\HeaderA{Cget\_inputs}{Returns inputs}{Cget.Rul.inputs}
%
\begin{Description}
Returns inputs
\end{Description}
%
\begin{Usage}
\begin{verbatim}
Cget_inputs()
\end{verbatim}
\end{Usage}
%
\begin{Value}
all inputs
\end{Value}
\HeaderA{Cget\_n\_events}{Returns total number of events.}{Cget.Rul.n.Rul.events}
%
\begin{Description}
Returns total number of events.
\end{Description}
%
\begin{Usage}
\begin{verbatim}
Cget_n_events()
\end{verbatim}
\end{Usage}
%
\begin{Value}
number of events
\end{Value}
\HeaderA{Cget\_output}{Main outputs of the current run.}{Cget.Rul.output}
%
\begin{Description}
Main outputs of the current run.
\end{Description}
%
\begin{Usage}
\begin{verbatim}
Cget_output()
\end{verbatim}
\end{Usage}
%
\begin{Value}
number of agents, cumulative time, number of deaths, number of COPD cases, as well as exacerbation statistics and QALYs.
\end{Value}
\HeaderA{Cget\_output\_ex}{Extra outputs from the model}{Cget.Rul.output.Rul.ex}
%
\begin{Description}
Extra outputs from the model
\end{Description}
%
\begin{Usage}
\begin{verbatim}
Cget_output_ex()
\end{verbatim}
\end{Usage}
%
\begin{Value}
Extra outputs from the model.
\end{Value}
\HeaderA{Cget\_runtime\_stats}{Returns run time stats.}{Cget.Rul.runtime.Rul.stats}
%
\begin{Description}
Returns run time stats.
\end{Description}
%
\begin{Usage}
\begin{verbatim}
Cget_runtime_stats()
\end{verbatim}
\end{Usage}
%
\begin{Value}
agent size as well as memory and random variable fill stats.
\end{Value}
\HeaderA{Cget\_settings}{Returns current settings.}{Cget.Rul.settings}
%
\begin{Description}
Returns current settings.
\end{Description}
%
\begin{Usage}
\begin{verbatim}
Cget_settings()
\end{verbatim}
\end{Usage}
%
\begin{Value}
current settings.
\end{Value}
\HeaderA{Cget\_smith}{Returns agent Smith.}{Cget.Rul.smith}
%
\begin{Description}
Returns agent Smith.
\end{Description}
%
\begin{Usage}
\begin{verbatim}
Cget_smith()
\end{verbatim}
\end{Usage}
%
\begin{Value}
agent smith.
\end{Value}
\HeaderA{Cset\_input\_var}{Sets input variables.}{Cset.Rul.input.Rul.var}
%
\begin{Description}
Sets input variables.
\end{Description}
%
\begin{Usage}
\begin{verbatim}
Cset_input_var(name, value)
\end{verbatim}
\end{Usage}
%
\begin{Arguments}
\begin{ldescription}
\item[\code{name}] a string

\item[\code{value}] a number
\end{ldescription}
\end{Arguments}
%
\begin{Value}
0 if successful
\end{Value}
\HeaderA{Cset\_settings\_var}{Sets model settings.}{Cset.Rul.settings.Rul.var}
%
\begin{Description}
Sets model settings.
\end{Description}
%
\begin{Usage}
\begin{verbatim}
Cset_settings_var(name, value)
\end{verbatim}
\end{Usage}
%
\begin{Arguments}
\begin{ldescription}
\item[\code{name}] a name

\item[\code{value}] a value
\end{ldescription}
\end{Arguments}
%
\begin{Value}
0 if successful.
\end{Value}
\HeaderA{estimate\_memory\_required}{Estimate memory required for simulation}{estimate.Rul.memory.Rul.required}
%
\begin{Description}
Estimate memory required for simulation
\end{Description}
%
\begin{Usage}
\begin{verbatim}
estimate_memory_required(n_agents, record_mode = 0, time_horizon = 20)
\end{verbatim}
\end{Usage}
%
\begin{Arguments}
\begin{ldescription}
\item[\code{n\_agents}] number of agents

\item[\code{record\_mode}] recording mode (0=none, 1=agent, 2=event, 3=some\_event)

\item[\code{time\_horizon}] simulation time horizon in years
\end{ldescription}
\end{Arguments}
%
\begin{Value}
estimated memory in bytes
\end{Value}
\HeaderA{export\_figures}{Runs the model and exports an excel file with all output data}{export.Rul.figures}
%
\begin{Description}
Runs the model and exports an excel file with all output data
\end{Description}
%
\begin{Usage}
\begin{verbatim}
export_figures(nPatients = 10000)
\end{verbatim}
\end{Usage}
%
\begin{Arguments}
\begin{ldescription}
\item[\code{nPatients}] number of agents
\end{ldescription}
\end{Arguments}
%
\begin{Value}
an excel file with all output
\end{Value}
\HeaderA{express\_matrix}{Express matrix.}{express.Rul.matrix}
%
\begin{Description}
Express matrix.
\end{Description}
%
\begin{Usage}
\begin{verbatim}
express_matrix(mtx)
\end{verbatim}
\end{Usage}
%
\begin{Arguments}
\begin{ldescription}
\item[\code{mtx}] a matrix
\end{ldescription}
\end{Arguments}
\HeaderA{get\_agent\_events}{Returns events specific to an agent.}{get.Rul.agent.Rul.events}
%
\begin{Description}
Returns events specific to an agent.
\end{Description}
%
\begin{Usage}
\begin{verbatim}
get_agent_events(id)
\end{verbatim}
\end{Usage}
%
\begin{Arguments}
\begin{ldescription}
\item[\code{id}] Agent number
\end{ldescription}
\end{Arguments}
%
\begin{Value}
dataframe consisting all events specific to agent \code{id}
\end{Value}
\HeaderA{get\_agent\_size\_bytes}{Get size of agent struct in bytes (from C code)}{get.Rul.agent.Rul.size.Rul.bytes}
%
\begin{Description}
Get size of agent struct in bytes (from C code)
\end{Description}
%
\begin{Usage}
\begin{verbatim}
get_agent_size_bytes()
\end{verbatim}
\end{Usage}
%
\begin{Value}
size in bytes
\end{Value}
\HeaderA{get\_all\_events}{Returns all events.}{get.Rul.all.Rul.events}
%
\begin{Description}
Returns all events.
\end{Description}
%
\begin{Usage}
\begin{verbatim}
get_all_events()
\end{verbatim}
\end{Usage}
%
\begin{Value}
dataframe consisting all events.
\end{Value}
\HeaderA{get\_available\_memory}{Get available system memory (platform-specific)}{get.Rul.available.Rul.memory}
%
\begin{Description}
Get available system memory (platform-specific)
\end{Description}
%
\begin{Usage}
\begin{verbatim}
get_available_memory()
\end{verbatim}
\end{Usage}
%
\begin{Value}
available memory in bytes
\end{Value}
\HeaderA{get\_default\_settings}{Exports default settings}{get.Rul.default.Rul.settings}
%
\begin{Description}
Exports default settings
\end{Description}
%
\begin{Usage}
\begin{verbatim}
get_default_settings()
\end{verbatim}
\end{Usage}
%
\begin{Value}
default settings
\end{Value}
\HeaderA{get\_errors}{Returns errors}{get.Rul.errors}
%
\begin{Description}
Returns errors
\end{Description}
%
\begin{Usage}
\begin{verbatim}
get_errors()
\end{verbatim}
\end{Usage}
%
\begin{Value}
a text with description of error messages
\end{Value}
\HeaderA{get\_events\_by\_type}{Returns certain events by type}{get.Rul.events.Rul.by.Rul.type}
%
\begin{Description}
Returns certain events by type
\end{Description}
%
\begin{Usage}
\begin{verbatim}
get_events_by_type(event_type)
\end{verbatim}
\end{Usage}
%
\begin{Arguments}
\begin{ldescription}
\item[\code{event\_type}] event\_type number
\end{ldescription}
\end{Arguments}
%
\begin{Value}
dataframe consisting all events of the type \code{event\_type}
\end{Value}
\HeaderA{get\_input}{Returns a list of default model input values}{get.Rul.input}
%
\begin{Description}
Returns a list of default model input values
\end{Description}
%
\begin{Usage}
\begin{verbatim}
get_input(
  age0 = 40,
  time_horizon = 20,
  discount_cost = 0,
  discount_qaly = 0.03,
  closed_cohort = 0,
  jurisdiction = "canada"
)
\end{verbatim}
\end{Usage}
%
\begin{Arguments}
\begin{ldescription}
\item[\code{age0}] Starting age in the model

\item[\code{time\_horizon}] Model time horizon

\item[\code{discount\_cost}] Discounting for cost outcomes

\item[\code{discount\_qaly}] Discounting for QALY outcomes

\item[\code{closed\_cohort}] Whether the model should run as closed\_cohort, open-population by default.

\item[\code{jurisdiction}] Jurisdiction for model parameters ("canada" or "us")
\end{ldescription}
\end{Arguments}
\HeaderA{get\_list\_elements}{Get list elements}{get.Rul.list.Rul.elements}
%
\begin{Description}
Get list elements
\end{Description}
%
\begin{Usage}
\begin{verbatim}
get_list_elements(ls, running_name = "")
\end{verbatim}
\end{Usage}
%
\begin{Arguments}
\begin{ldescription}
\item[\code{ls}] ls

\item[\code{running\_name}] running\_name
\end{ldescription}
\end{Arguments}
\HeaderA{get\_sample\_output}{Returns a sample output for a given year and gender.}{get.Rul.sample.Rul.output}
%
\begin{Description}
Returns a sample output for a given year and gender.
\end{Description}
%
\begin{Usage}
\begin{verbatim}
get_sample_output(year, sex)
\end{verbatim}
\end{Usage}
%
\begin{Arguments}
\begin{ldescription}
\item[\code{year}] a number

\item[\code{sex}] a number, 0 for male and 1 for female
\end{ldescription}
\end{Arguments}
%
\begin{Value}
that specific output
\end{Value}
\HeaderA{init\_session}{Initializes a model. Allocates memory to the C engine.}{init.Rul.session}
%
\begin{Description}
Initializes a model. Allocates memory to the C engine.
\end{Description}
%
\begin{Usage}
\begin{verbatim}
init_session(settings = get_default_settings())
\end{verbatim}
\end{Usage}
%
\begin{Arguments}
\begin{ldescription}
\item[\code{settings}] customized settings.
\end{ldescription}
\end{Arguments}
%
\begin{Value}
0 if successful.
\end{Value}
\HeaderA{mvrnormArma}{Samples from a multivariate normal}{mvrnormArma}
%
\begin{Description}
Samples from a multivariate normal
\end{Description}
%
\begin{Usage}
\begin{verbatim}
mvrnormArma(n, mu, sigma)
\end{verbatim}
\end{Usage}
%
\begin{Arguments}
\begin{ldescription}
\item[\code{n}] number of samples to be taken

\item[\code{mu}] the mean

\item[\code{sigma}] the covariance matrix
\end{ldescription}
\end{Arguments}
%
\begin{Value}
the multivariate normal sample
\end{Value}
\HeaderA{report\_COPD\_by\_ctime}{Reports COPD related stats.}{report.Rul.COPD.Rul.by.Rul.ctime}
%
\begin{Description}
Reports COPD related stats.
\end{Description}
%
\begin{Usage}
\begin{verbatim}
report_COPD_by_ctime(n_sim = 10^6)
\end{verbatim}
\end{Usage}
%
\begin{Arguments}
\begin{ldescription}
\item[\code{n\_sim}] number of simulated agents.
\end{ldescription}
\end{Arguments}
%
\begin{Value}
COPD-related stats
\end{Value}
\HeaderA{report\_exacerbation\_by\_time}{Reports exacerbation-related stats.}{report.Rul.exacerbation.Rul.by.Rul.time}
%
\begin{Description}
Reports exacerbation-related stats.
\end{Description}
%
\begin{Usage}
\begin{verbatim}
report_exacerbation_by_time(n_sim = 10^5)
\end{verbatim}
\end{Usage}
%
\begin{Arguments}
\begin{ldescription}
\item[\code{n\_sim}] number of simulated agents.
\end{ldescription}
\end{Arguments}
%
\begin{Value}
exacerbation-related stats
\end{Value}
\HeaderA{resume}{Resumes running of model.}{resume}
%
\begin{Description}
Resumes running of model.
\end{Description}
%
\begin{Usage}
\begin{verbatim}
resume(max_n_agents = NULL)
\end{verbatim}
\end{Usage}
%
\begin{Arguments}
\begin{ldescription}
\item[\code{max\_n\_agents}] maximum number of agents
\end{ldescription}
\end{Arguments}
%
\begin{Value}
0 if successful.
\end{Value}
\HeaderA{run}{Runs the model, after a session has been initialized.}{run}
%
\begin{Description}
Runs the model, after a session has been initialized.
\end{Description}
%
\begin{Usage}
\begin{verbatim}
run(max_n_agents = NULL, input = NULL)
\end{verbatim}
\end{Usage}
%
\begin{Arguments}
\begin{ldescription}
\item[\code{max\_n\_agents}] maximum number of agents

\item[\code{input}] customized input criteria
\end{ldescription}
\end{Arguments}
%
\begin{Value}
0 if successful.
\end{Value}
\HeaderA{sanity\_check}{Basic tests of model functionality. Serious issues if the test does not pass.}{sanity.Rul.check}
%
\begin{Description}
Basic tests of model functionality. Serious issues if the test does not pass.
\end{Description}
%
\begin{Usage}
\begin{verbatim}
sanity_check()
\end{verbatim}
\end{Usage}
%
\begin{Value}
tests results
\end{Value}
\HeaderA{sanity\_COPD}{Basic COPD test.}{sanity.Rul.COPD}
%
\begin{Description}
Basic COPD test.
\end{Description}
%
\begin{Usage}
\begin{verbatim}
sanity_COPD()
\end{verbatim}
\end{Usage}
%
\begin{Value}
validation test results
\end{Value}
\HeaderA{terminate\_session}{Terminates a session and releases allocated memory.}{terminate.Rul.session}
%
\begin{Description}
Terminates a session and releases allocated memory.
\end{Description}
%
\begin{Usage}
\begin{verbatim}
terminate_session()
\end{verbatim}
\end{Usage}
%
\begin{Value}
0 if successful.
\end{Value}
\HeaderA{test\_case\_detection}{Returns results of Case Detection strategies}{test.Rul.case.Rul.detection}
%
\begin{Description}
Returns results of Case Detection strategies
\end{Description}
%
\begin{Usage}
\begin{verbatim}
test_case_detection(
  n_sim = 10000,
  p_of_CD = 0.1,
  min_age = 40,
  min_pack_years = 0,
  only_smokers = 0,
  CD_method = "CDQ195"
)
\end{verbatim}
\end{Usage}
%
\begin{Arguments}
\begin{ldescription}
\item[\code{n\_sim}] number of agents

\item[\code{p\_of\_CD}] probability of recieving case detection given that an agent meets the selection criteria

\item[\code{min\_age}] minimum age that can recieve case detection

\item[\code{min\_pack\_years}] minimum pack years that can recieve case detection

\item[\code{only\_smokers}] set to 1 if only smokers should recieve case detection

\item[\code{CD\_method}] Choose one case detection method: CDQ195", "CDQ165", "FlowMeter", "FlowMeter\_CDQ"
\end{ldescription}
\end{Arguments}
%
\begin{Value}
results of case detection strategy compared to no case detection
\end{Value}
\HeaderA{validate\_COPD}{Returns results of validation tests for COPD}{validate.Rul.COPD}
%
\begin{Description}
This function runs validation tests for COPD model outputs. It estimates the baseline
prevalence and incidence of COPD, along with sex-specific baseline COPD prevalence.
Additionally, it calculates the baseline prevalence of COPD by age groups and
smoking pack-years. It also estimates the coefficients for the relationships between
age, pack-years, smoking status, and the prevalence of COPD.
\end{Description}
%
\begin{Usage}
\begin{verbatim}
validate_COPD(incident_COPD_k = 1, return_CI = FALSE)
\end{verbatim}
\end{Usage}
%
\begin{Arguments}
\begin{ldescription}
\item[\code{incident\_COPD\_k}] a number (default=1) by which the incidence rate of COPD will be multiplied.

\item[\code{return\_CI}] if TRUE, returns 95 percent confidence intervals for the "Year" coefficient
\end{ldescription}
\end{Arguments}
%
\begin{Value}
validation test results
\end{Value}
\HeaderA{validate\_diagnosis}{Returns results of validation tests for diagnosis}{validate.Rul.diagnosis}
%
\begin{Description}
This function returns a table showing the proportion of COPD patients diagnosed
over the model's runtime. It also provides a second table that breaks down the proportion
of diagnosed patients by COPD severity. Additionally, the function generates a plot
to visualize the distribution of diagnoses over time.
\end{Description}
%
\begin{Usage}
\begin{verbatim}
validate_diagnosis(n_sim = 10000)
\end{verbatim}
\end{Usage}
%
\begin{Arguments}
\begin{ldescription}
\item[\code{n\_sim}] number of agents
\end{ldescription}
\end{Arguments}
%
\begin{Value}
validation test results
\end{Value}
\HeaderA{validate\_exacerbation}{Returns results of validation tests for exacerbation rates}{validate.Rul.exacerbation}
%
\begin{Description}
This function runs validation tests for COPD exacerbation rates by GOLD stage
and compares them with reference values from studies such as CanCOLD, CIHI,
and Hoogendoorn. It simulates exacerbation events, and returns key metrics,
including overall exacerbation rates, rates by GOLD stage, and rates in
diagnosed vs. undiagnosed patients.
\end{Description}
%
\begin{Usage}
\begin{verbatim}
validate_exacerbation(base_agents = 10000, input = NULL)
\end{verbatim}
\end{Usage}
%
\begin{Arguments}
\begin{ldescription}
\item[\code{base\_agents}] Number of agents in the simulation. Default is 1e4.

\item[\code{input}] EPIC inputs
\end{ldescription}
\end{Arguments}
%
\begin{Value}
validation test results
\end{Value}
\HeaderA{validate\_gpvisits}{Returns results of validation tests for GP visits}{validate.Rul.gpvisits}
%
\begin{Description}
This function returns Average number of GP visits by sex, COPD severity and
COPD diagnosis status along with their plots.
\end{Description}
%
\begin{Usage}
\begin{verbatim}
validate_gpvisits(n_sim = 10000)
\end{verbatim}
\end{Usage}
%
\begin{Arguments}
\begin{ldescription}
\item[\code{n\_sim}] number of agents
\end{ldescription}
\end{Arguments}
%
\begin{Value}
validation test results
\end{Value}
\HeaderA{validate\_lung\_function}{Returns results of validation tests for lung function}{validate.Rul.lung.Rul.function}
%
\begin{Description}
This function evaluates FEV1 (Forced Expiratory Volume in 1 second) values
and GOLD stage distributions to assess lung function in simulated patients.
\end{Description}
%
\begin{Usage}
\begin{verbatim}
validate_lung_function()
\end{verbatim}
\end{Usage}
%
\begin{Value}
validation test results
\end{Value}
\HeaderA{validate\_medication}{Returns results of validation tests for medication module.}{validate.Rul.medication}
%
\begin{Description}
This function returns plots showing medication usage over time
\end{Description}
%
\begin{Usage}
\begin{verbatim}
validate_medication(n_sim = 50000)
\end{verbatim}
\end{Usage}
%
\begin{Arguments}
\begin{ldescription}
\item[\code{n\_sim}] number of agents
\end{ldescription}
\end{Arguments}
%
\begin{Value}
validation test results for medication
\end{Value}
\HeaderA{validate\_mortality}{Returns results of validation tests for mortality rate}{validate.Rul.mortality}
%
\begin{Description}
This function returns a table and a plot of the difference between simulated and expected
(life table) mortality, by sex and age.
\end{Description}
%
\begin{Usage}
\begin{verbatim}
validate_mortality(
  n_sim = 5e+05,
  bgd = 1,
  bgd_h = 1,
  manual = 1,
  exacerbation = 1
)
\end{verbatim}
\end{Usage}
%
\begin{Arguments}
\begin{ldescription}
\item[\code{n\_sim}] number of simulated agents

\item[\code{bgd}] a number

\item[\code{bgd\_h}] a number

\item[\code{manual}] a number

\item[\code{exacerbation}] a number
\end{ldescription}
\end{Arguments}
%
\begin{Value}
validation test results
\end{Value}
\HeaderA{validate\_overdiagnosis}{Returns results of validation tests for overdiagnosis}{validate.Rul.overdiagnosis}
%
\begin{Description}
This function returns the proportion of non-COPD subjects overdiagnosed
over model time.
\end{Description}
%
\begin{Usage}
\begin{verbatim}
validate_overdiagnosis(n_sim = 10000)
\end{verbatim}
\end{Usage}
%
\begin{Arguments}
\begin{ldescription}
\item[\code{n\_sim}] number of agents
\end{ldescription}
\end{Arguments}
%
\begin{Value}
validation test results
\end{Value}
\HeaderA{validate\_payoffs}{Returns results of validation tests for payoffs, costs and QALYs}{validate.Rul.payoffs}
%
\begin{Description}
Returns results of validation tests for payoffs, costs and QALYs
\end{Description}
%
\begin{Usage}
\begin{verbatim}
validate_payoffs(
  nPatient = 1e+06,
  disableDiscounting = TRUE,
  disableExacMortality = TRUE
)
\end{verbatim}
\end{Usage}
%
\begin{Arguments}
\begin{ldescription}
\item[\code{nPatient}] number of simulated patients. Default is 1e6.

\item[\code{disableDiscounting}] if TRUE, discounting will be disabled for cost and QALY calculations. Default: TRUE

\item[\code{disableExacMortality}] if TRUE, mortality due to exacerbations will be disabled for cost and QALY calculations. Default: TRUE
\end{ldescription}
\end{Arguments}
%
\begin{Value}
validation test results
\end{Value}
\HeaderA{validate\_population}{Returns simulated vs. predicted population table and a plot}{validate.Rul.population}
%
\begin{Description}
Returns simulated vs. predicted population table and a plot
\end{Description}
%
\begin{Usage}
\begin{verbatim}
validate_population(remove_COPD = 0, incidence_k = 1, savePlots = 0)
\end{verbatim}
\end{Usage}
%
\begin{Arguments}
\begin{ldescription}
\item[\code{remove\_COPD}] 0 or 1, indicating whether COPD-caused mortality should be removed

\item[\code{incidence\_k}] a number (default=1) by which the incidence rate of population will be multiplied.

\item[\code{savePlots}] 0 or 1, exports 300 DPI population growth and pyramid plots comparing simulated vs. predicted population
\end{ldescription}
\end{Arguments}
%
\begin{Value}
returns a table showing predicted (StatsCan) and simulated population values
\end{Value}
\HeaderA{validate\_smoking}{Returns results of validation tests for smoking module.}{validate.Rul.smoking}
%
\begin{Description}
It compares simulated smoking prevalence and trends against observed data to
assess the model's accuracy.
\end{Description}
%
\begin{Usage}
\begin{verbatim}
validate_smoking(remove_COPD = 1, intercept_k = NULL)
\end{verbatim}
\end{Usage}
%
\begin{Arguments}
\begin{ldescription}
\item[\code{remove\_COPD}] 0 or 1. whether to remove COPD-related mortality.

\item[\code{intercept\_k}] a number
\end{ldescription}
\end{Arguments}
%
\begin{Value}
validation test results
\end{Value}
\HeaderA{validate\_survival}{Returns the Kaplan Meier curve comparing COPD and non-COPD}{validate.Rul.survival}
%
\begin{Description}
Returns the Kaplan Meier curve comparing COPD and non-COPD
\end{Description}
%
\begin{Usage}
\begin{verbatim}
validate_survival(savePlots = FALSE, base_agents = 10000)
\end{verbatim}
\end{Usage}
%
\begin{Arguments}
\begin{ldescription}
\item[\code{savePlots}] TRUE or FALSE (default), exports 300 DPI population growth and pyramid plots comparing simulated vs. predicted population

\item[\code{base\_agents}] Number of agents in the simulation. Default is 1e4.
\end{ldescription}
\end{Arguments}
%
\begin{Value}
validation test results
\end{Value}
\HeaderA{validate\_symptoms}{Returns results of validation tests for Symptoms}{validate.Rul.symptoms}
%
\begin{Description}
This function plots the prevalence of cough, dyspnea, phlegm and wheeze
over time and by GOLD stage.
\end{Description}
%
\begin{Usage}
\begin{verbatim}
validate_symptoms(n_sim = 10000)
\end{verbatim}
\end{Usage}
%
\begin{Arguments}
\begin{ldescription}
\item[\code{n\_sim}] number of agents
\end{ldescription}
\end{Arguments}
%
\begin{Value}
validation test results
\end{Value}
\HeaderA{validate\_treatment}{Returns results of validation tests for Treatment}{validate.Rul.treatment}
%
\begin{Description}
This function runs validation tests to examine how treatment initiated at diagnosis
influences exacerbation rates in COPD patients. It compares exacerbation rates between
diagnosed and undiagnosed patients and assesses the impact of treatment.
\end{Description}
%
\begin{Usage}
\begin{verbatim}
validate_treatment(n_sim = 10000)
\end{verbatim}
\end{Usage}
%
\begin{Arguments}
\begin{ldescription}
\item[\code{n\_sim}] number of agents
\end{ldescription}
\end{Arguments}
%
\begin{Value}
validation test results
\end{Value}
\printindex{}
\end{document}
